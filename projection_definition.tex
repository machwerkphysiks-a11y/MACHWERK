\section{Projection}

\subsection{Definition}

A projection is a mapping
\[
\Pi : \mathcal{U} \rightarrow \mathcal{M}
\]
from the relational reference domain $\mathcal{U}$
into an observable or computable domain $\mathcal{M}$.

$\Pi$ is not assumed to preserve structure,
distance, dimensionality, or separability.

\subsection{Properties}

In general, $\Pi$ is:
\begin{itemize}
    \item information-reducing,
    \item context-dependent,
    \item not globally injective,
    \item not required to be surjective.
\end{itemize}

\subsection{Injectivity}

Injectivity is a local property.

$\Pi$ is injective on a subset $A \subset \mathcal{U}$ if:
\[
\forall u_1,u_2 \in A:
\Pi(u_1)=\Pi(u_2)\Rightarrow u_1=u_2
\]

Only in injective regions
does a unique physical interpretation exist.

\subsection{Loss of injectivity}

The loss of injectivity marks
the transition from $m_2$ to $m_3$.

This transition is not a physical event,
but a structural property of the projection.

\subsection{Interpretational consequence}

Projection does not create reality.

It limits access to relational structure.

Different relational states in $\mathcal{U}$
may appear identical after projection.

This ambiguity is formal,
not ontological.
