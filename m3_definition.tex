\section{Definition of $m_3$}

\subsection{Conceptual role}

$m_3$ denotes the projection domain
in which relational expressions
remain formally computable
but are no longer uniquely back-referencable
to $\mathcal{U}$.

$m_3$ is not a physical space
and does not represent an extension of reality.

It is a \emph{formal continuation domain}.

\subsection{Formal definition}

Let
\[
\Pi : \mathcal{U} \rightarrow m
\]
be a projection.

Then $m_3$ is defined as:
\[
m_3 := \{\, u \in \mathcal{U} \mid \Pi \text{ is not injective at } u \,\}
\]

Equivalently:
\[
u \in m_3
\;\Longleftrightarrow\;
\exists\, u_1 \neq u_2 \in \mathcal{U} :
\Pi(u_1) = \Pi(u_2)
\]

\subsection{Interpretational status}

Statements evaluated in $m_3$
possess no unique physical origin.

They cannot be assigned
a determinate relational cause.

Nevertheless,
formal consistency is preserved.

\subsection{Function within the framework}

$m_3$ serves three purposes:
\begin{itemize}
    \item marking the limit of physical interpretability,
    \item allowing controlled extrapolation,
    \item preventing implicit ontological extension.
\end{itemize}

\subsection{Relation to $m_2$}

$m_2$ and $m_3$ are complementary:
\[
m_2 \cup m_3 = \Pi(\mathcal{U})
\quad\text{and}\quad
m_2 \cap m_3 = \emptyset
\]

The transition from $m_2$ to $m_3$
is governed by the Schwarzgrenze $\Sigma$.
