\section{Bell Admissibility Test}

This test examines Bell-type correlation statements
under non-injective projection.

\subsection{Setup}

Let
\[
\Pi : \mathcal{U} \rightarrow m_2
\]
be a projection into an observable calculation domain.

Assume two relational configurations
\[
u_1 \neq u_2 \in \mathcal{U}
\]
such that
\[
\Pi(u_1) = \Pi(u_2).
\]

These configurations lie within the black boundary
\[
u_1, u_2 \in \Sigma.
\]

\subsection{Bell-Type Statement}

Consider a correlation statement
\[
C(u)
\]
that assigns measurement outcomes to underlying relational states.

Standard Bell formulations implicitly assume
that distinct underlying states correspond to distinct observables.

\subsection{Admissibility Condition}

Under the CRA axiom, a statement is physically admissible only if:
\[
\Pi(u_1) = \Pi(u_2)
\;\Rightarrow\;
C(u_1) = C(u_2).
\]

If the correlation depends on distinctions between
$u_1$ and $u_2$ that are not observable in $m_2$,
the statement is physically inadmissible.

\subsection{Result}

Bell-type statistical correlations remain formally valid.

However, any statement asserting
distinct physical realities for non-injectively projected states
is inadmissible under CRA.

\subsection{Conclusion}

The test does not invalidate quantum statistics.

It restricts the physical interpretation of correlations
to projection-invariant relational content.
