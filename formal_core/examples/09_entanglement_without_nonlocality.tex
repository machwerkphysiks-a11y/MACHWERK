\section{Entanglement Without Nonlocality}

This application demonstrates
how quantum entanglement
can be represented
without invoking nonlocal interaction.

\subsection{Setup}

Let
\[
u \in \mathcal{U}
\]
be a relational configuration
containing a minimal TRD structure.

Assume a projection
\[
\Pi : \mathcal{U} \rightarrow m_2
\]
into an observable calculation space.

\subsection{Relational Unity}

In $\mathcal{U}$,
the relational configuration is single and undivided.

There is no spatial separation,
no distance,
and no signal propagation.

\subsection{Projection Separation}

Under projection,
the unified relational structure
appears as two distinct observable subsystems
in $m_2$.

This separation is a projection artifact,
not a physical division.

\subsection{Measurement Correlation}

A measurement corresponds
to selecting a projection slice.

Observed correlations arise
because both observables
originate from the same relational configuration.

\subsection{No Information Transfer}

No information travels between subsystems.

Correlation is established
prior to projection
and does not require causal signaling.

\subsection{Black Boundary Role}

At the projection boundary,
multiple relational states
map to identical observables.

Statistical behavior emerges
from non-injective projection,
not from hidden dynamics.

\subsection{Result}

Quantum entanglement is explained
as a consequence of
relational unity and projection,
without nonlocal interaction.
