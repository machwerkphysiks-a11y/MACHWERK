\section{Quantum Correlation Without Signaling}

This application demonstrates
how quantum correlations arise
without information transfer
within the MACHWERK framework.

\subsection{Setup}

Let
\[
\Pi : \mathcal{U} \rightarrow m_2
\]
be a projection into an observable domain.

Assume a single relational configuration
\[
u \in \mathcal{U}
\]
containing multiple internally correlated relations.

\subsection{Relational Unity}

In $\mathcal{U}$,
there is no spatial separation.
Relational components of $u$
are not localized or ordered by distance.

What appears as two separated systems
in $m_2$
corresponds to one unified relational structure in $\mathcal{U}$.

\subsection{Projection Multiplicity}

Under projection,
distinct observable events
\[
o_A, o_B \in m_2
\]
may originate from the same
relational source $u$.

The appearance of separation
is introduced entirely by $\Pi$.

\subsection{Correlation Mechanism}

Observed correlations arise because
$o_A$ and $o_B$
are projections of the same relational configuration.

No signal propagates between them.
No causal influence is exchanged in $m_2$.

\subsection{CRA Admissibility}

The correlation is physically admissible
because it depends only on
projected relations,
not on inaccessible distinctions within $\mathcal{U}$.

Statements attempting to distinguish
multiple preimages of $\Pi$
are inadmissible.

\subsection{Bell-Type Constraints}

Statistical correlations may violate
classical locality bounds
without violating causality.

This reflects projection structure,
not superluminal communication.

\subsection{Result}

Quantum correlation is modeled
as shared relational origin
combined with non-injective projection,
fully consistent with
no-signaling requirements.
