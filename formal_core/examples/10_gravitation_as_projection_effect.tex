\section{Gravitation as a Projection Effect}

This application shows
how gravitational interaction
can be treated as a projection-dependent effect
rather than a fundamental force.

\subsection{Setup}

Let
\[
\Pi : \mathcal{U} \rightarrow m_2
\]
be a projection into an observable calculation space.

Assume a relational configuration
\[
u \in \mathcal{U}
\]
with varying relational densities.

\subsection{Relational Gradient}

In $\mathcal{U}$,
no force exists.

Only relational density
and relational imbalance
between process configurations are defined.

\subsection{Projection into Geometry}

Under projection,
relational gradients
appear as geometric curvature
in $m_2$.

This curvature is not intrinsic,
but induced by projection.

\subsection{Apparent Force}

Observed acceleration
results from maintaining relational consistency
under projection constraints.

No force is transmitted.
No interaction propagates.

\subsection{Black Boundary Interpretation}

Near projection boundaries,
relational collapse increases.

This leads to extreme curvature effects,
commonly interpreted as strong gravitation.

\subsection{No Ontological Commitment}

This formulation does not assert
what gravitation \emph{is},
only how it becomes observable.

\subsection{Result}

Gravitation appears
as a projection artifact
of relational structure,
not as a fundamental interaction.
