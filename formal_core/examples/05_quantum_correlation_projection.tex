\section{Quantum Correlation as Projection Effect}

This application demonstrates how quantum correlations
can be treated as projection effects
without invoking superluminal signaling.

\subsection{Setup}

Let a relational configuration
\[
u \in \mathcal{U}
\]
encode a closed relational structure
that cannot be decomposed into independent subsystems
within the observable domain.

Assume a projection
\[
\Pi : \mathcal{U} \rightarrow m_2
\]
into an observable calculation space.

\subsection{Non-Separability}

Within $\mathcal{U}$,
the configuration $u$ is single and undivided.
No spatial separation is defined at this level.

Spatial distinction arises only after projection.

\subsection{Projection Outcome}

Under $\Pi$,
the relational configuration appears as
two spatially separated observables
\[
\Pi(u) \mapsto (A, B)
\]
within $m_2$.

The separation is a property of the projection,
not of the underlying relation.

\subsection{Correlation}

Measured correlations between $A$ and $B$
reflect constraints already present in $u$.

No information transfer occurs within $m_2$.
All consistency is inherited from $\mathcal{U}$.

\subsection{Boundary Aspect}

The projection is non-injective:
distinct relational aspects of $u$
collapse onto identical observable outcomes.

This places the configuration at or near
the black boundary.

\subsection{Admissibility}

Observable correlations are admissible
only as functions of projected observables.

Statements about internal structure of $u$
are formally definable
but not physically interpretable.

\subsection{Result}

Quantum correlation appears as non-local
only when interpreted within $m_2$.

Formally, it is a local consistency condition
within $\mathcal{U}$
revealed through a non-injective projection.
