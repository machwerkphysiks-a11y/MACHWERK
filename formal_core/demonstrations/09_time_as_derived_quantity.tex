\section{Time as a Derived Quantity}

\subsection{Problem Class}

Time is traditionally treated as a fundamental dimension
with universal ordering and metric structure.

Within the MACHWERK framework,
time is not fundamental.

Time is a derived relational construct.

\subsection{Absence of Time in $\mathcal{U}$}

The relational full space $\mathcal{U}$ contains:

\begin{itemize}
\item no global time parameter
\item no universal ordering
\item no intrinsic temporal metric
\end{itemize}

Relations in $\mathcal{U}$ are atemporal.
They do not evolve.
They are compared.

\subsection{Emergence via Process Comparison}

Let
\[
\Pi_t : \mathcal{U} \rightarrow m_2
\]
be a projection into a domain where
stable rate comparisons exist.

Time emerges as an ordering parameter
derived from ratios of process rates:
\[
t \sim \frac{R_i}{R_j}
\]

Time measures relational change,
not absolute flow.

\subsection{Relational Ordering}

Temporal ordering is valid only within
a fixed projection context.

Two relational configurations
\[
u_1, u_2 \in \mathcal{U}
\]
may admit an ordering under $\Pi_t$,
even though no ordering exists in $\mathcal{U}$ itself.

Causality is therefore projection-relative.

\subsection{Local Validity}

Time is locally well-defined
only where process ratios remain stable.

Different projection domains
may induce incompatible time parameters
without contradiction.

This formally accommodates:

\begin{itemize}
\item time dilation
\item relativistic simultaneity loss
\item observer-dependent temporal order
\end{itemize}

without introducing paradoxes.

\subsection{Black Boundary for Time}

Define
\[
\Sigma_t := \{ u \in \mathcal{U} \mid \Pi_t \text{ is non-injective} \}
\]

Beyond $\Sigma_t$:

\begin{itemize}
\item process relations remain consistent
\item temporal ordering becomes ambiguous
\item time-based interpretation fails
\end{itemize}

Equations may still be extended,
but temporal meaning is no longer admissible.

\subsection{CRA Constraint}

A time-related statement $S(t)$ is admissible
only if it depends exclusively on projected
process ratios and remains invariant
under substitutions preserving $\Pi_t$.

Statements assuming absolute or global time
violate CRA.

\subsection{Formal Conclusion}

Time in MACHWERK is not a primitive dimension.

It is a projection-dependent ordering parameter
emerging from relational process comparison.

Beyond its black boundary,
time remains formally calculable
but loses physical interpretability.
