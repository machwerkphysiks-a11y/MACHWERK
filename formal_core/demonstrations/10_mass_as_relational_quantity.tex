\section{Mass as a Relational Quantity}

\subsection{Problem Class}

In classical and relativistic physics,
mass is treated as a primitive physical property.

It is introduced as an intrinsic attribute
of particles or systems.

Within MACHWERK,
mass is not fundamental.

Mass is a relational projection effect.

\subsection{Absence of Mass in $\mathcal{U}$}

The relational full space $\mathcal{U}$ contains:

\begin{itemize}
\item no intrinsic mass values
\item no particle ontology
\item no absolute inertial property
\end{itemize}

Relations in $\mathcal{U}$ do not carry mass.
They carry structure.

\subsection{Emergence via Process Resistance}

Let
\[
\Pi_m : \mathcal{U} \rightarrow m_2
\]
be a projection into a domain where
process rates and reactions are comparable.

Mass emerges as a measure of
relational resistance to process change.

Formally,
\[
m \sim \frac{R_{\text{external}}}{R_{\text{response}}}
\]

Mass quantifies relational inertia,
not substance.

\subsection{Relational Interpretation}

A system appears massive
if its internal relational configuration
changes slowly relative to external interactions.

Different relational environments
yield different effective masses
without contradiction.

Mass is therefore context-dependent,
but locally stable.

\subsection{Inertia without Ontology}

Inertial behavior does not require
an intrinsic carrier.

It arises from relational coupling
between a system and its surrounding
process network.

This recovers Mach-type behavior
without invoking a preferred background.

\subsection{Black Boundary for Mass}

Define
\[
\Sigma_m := \{ u \in \mathcal{U} \mid \Pi_m \text{ is non-injective} \}
\]

Beyond $\Sigma_m$:

\begin{itemize}
\item resistance remains formally definable
\item multiple relational configurations
      project to identical mass values
\item mass loses unique interpretability
\end{itemize}

The calculation survives.
The physical meaning does not.

\subsection{CRA Constraint}

A mass-related statement $S(m)$ is admissible
only if it depends solely on projected
rate ratios and is invariant under
relational substitutions preserving $\Pi_m$.

Any statement assuming mass as intrinsic
violates CRA.

\subsection{Formal Conclusion}

Mass in MACHWERK is not a property of objects.

It is a projection-dependent measure
of relational resistance within a given domain.

Beyond its black boundary,
mass remains calculable
but ceases to be physically interpretable.
