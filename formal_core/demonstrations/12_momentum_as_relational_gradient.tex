\section{Momentum as a Relational Gradient}

\subsection{Problem Class}

Momentum is traditionally defined
as mass times velocity.

Within MACHWERK,
neither mass nor velocity are fundamental.

\subsection{Absence in $\mathcal{U}$}

The relational space $\mathcal{U}$ contains:

\begin{itemize}
\item no trajectories
\item no velocities
\item no inertial frames
\end{itemize}

Momentum does not exist at the relational level.

\subsection{Projection Definition}

Let
\[
\Pi_p : \mathcal{U} \rightarrow m_2
\]
be a projection introducing ordered comparison
between relational states.

Momentum emerges as a gradient:
\[
p \sim \frac{\Delta R}{\Delta \lambda}
\]

It measures directional change
of relational configurations.

\subsection{Interpretation}

Momentum is not motion.
It is relational asymmetry
under ordered comparison.

\subsection{Black Boundary}

At the momentum black boundary,
directional distinctions collapse.
Momentum remains computable
but loses reconstructive meaning.

\subsection{CRA Constraint}

Only momentum expressions invariant
under non-injective substitution
are physically admissible.

\subsection{Conclusion}

Momentum is a projected relational gradient,
not an intrinsic property of objects.
