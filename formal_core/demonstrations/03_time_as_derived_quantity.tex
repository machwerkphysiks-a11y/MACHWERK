\section{Time as a Derived Quantity}

\subsection{Problem Class}

Time is commonly treated as a fundamental background parameter.
Within the MACHWERK framework, time is analyzed as a
\emph{derived and projected quantity},
not as a primitive element of description.

The core question is not how time flows,
but under which conditions time statements are admissible.

\subsection{Relational Background}

In the relational full space $\mathcal{U}$,
there is no distinguished time variable.

Relations exist without ordering,
duration, or temporal metric.
There are only configurations and changes
expressed as relational differences.

Time does not exist in $\mathcal{U}$.

\subsection{Projection into Observable Domains}

Let
\[
\Pi_t : \mathcal{U} \rightarrow m_2
\]
be a projection into an observable calculation domain
where time-like parameters are defined.

Operationally, time emerges from comparative process rates:
\[
t \sim \frac{\Delta N}{R}
\]
where $\Delta N$ denotes counted transitions
and $R$ a reference process rate.

\subsection{Context Dependence}

Different choices of reference processes
lead to different projected time parameters.

Thus, distinct relational configurations
\[
u_1 \neq u_2 \in \mathcal{U}
\]
may satisfy
\[
\Pi_t(u_1) = \Pi_t(u_2)
\]
even though their internal relational structures differ.

Time projection is therefore context-dependent
and not globally injective.

\subsection{Black Boundary for Time}

Define
\[
\Sigma_t := \{ u \in \mathcal{U} \mid \Pi_t \text{ is non-injective} \}
\]

Within $\Sigma_t$, temporal ordering remains calculable,
but it cannot be uniquely traced back
to a single relational configuration.

Time loses reconstructive power.

\subsection{CRA Admissibility}

A temporal statement $S(t)$ is admissible
only if it depends solely on projected observables
and remains invariant across all relational states
sharing the same projected time value.

Statements assigning absolute or universal status to time
violate the CRA axiom.

\subsection{Formal Conclusion}

Time in MACHWERK is not a fundamental dimension.

It is a projection-dependent ordering parameter
that arises from relational comparison.

Time is valid where projections are stable,
and becomes formally ambiguous
beyond the black boundary.
