\section{Mass as a Relational Quantity}

\subsection{Problem Class}

Mass is traditionally treated as an intrinsic property of an object.
Within the MACHWERK framework, mass is not intrinsic.

Mass is a relationally inferred quantity.

The framework does not redefine mass,
but constrains when mass statements are admissible.

\subsection{Absence of Intrinsic Mass in $\mathcal{U}$}

In the relational full space $\mathcal{U}$:

\begin{itemize}
\item no object carries intrinsic mass
\item no absolute inertia exists
\item no privileged reference background is defined
\end{itemize}

Only relations between processes and interactions exist.

Mass is not stored in $\mathcal{U}$.

\subsection{Emergence via Projection}

Let
\[
\Pi_m : \mathcal{U} \rightarrow m_2
\]
be a projection into an observable calculation domain
where comparative resistance to interaction is defined.

Mass emerges as a relational coefficient
linking force-like relations to acceleration-like relations:
\[
m \sim \frac{R_{\text{interaction}}}{R_{\text{response}}}
\]

Mass is therefore inferred,
not measured directly.

\subsection{Relational Interpretation}

A mass value encodes how a system participates
in relational change relative to others.

It does not describe an isolated property,
but a position within a network of relations.

Different relational configurations may yield
identical projected mass values:
\[
\Pi_m(u_1) = \Pi_m(u_2)
\quad \text{with} \quad u_1 \neq u_2
\]

\subsection{Black Boundary for Mass}

Define
\[
\Sigma_m := \{ u \in \mathcal{U} \mid \Pi_m \text{ is non-injective} \}
\]

Beyond $\Sigma_m$:

\begin{itemize}
\item mass values remain computable
\item inertia-like behavior persists
\item unique relational origin is lost
\end{itemize}

Mass continues to function numerically,
but no longer uniquely identifies relational structure.

\subsection{CRA Constraint}

A mass statement $S(m)$ is admissible
only if it depends exclusively on projected relations
and remains invariant under relational substitutions
that preserve $\Pi_m$.

Claims of intrinsic or absolute mass
are physically inadmissible under CRA.

\subsection{Formal Conclusion}

Mass in MACHWERK is not a substance
and not a stored property.

It is a projection-stable relational coefficient
valid only within injective domains.

Beyond the black boundary,
mass remains calculable but loses ontological exclusivity.
