\section{Schwarzgrenze}

\subsection{Definition}

The \emph{Schwarzgrenze} $\Sigma$ denotes the structural boundary
at which a projection loses injectivity.

Formally, let
\[
\Pi : \mathcal{U} \rightarrow m
\]
be a projection.

Then the Schwarzgrenze is defined as:
\[
\Sigma := \{\, u \in \mathcal{U} \mid \Pi \text{ is not injective at } u \,\}
\]

At $\Sigma$, distinct relational configurations
become indistinguishable under projection.

\subsection{Logical meaning}

Crossing the Schwarzgrenze does not invalidate formal expressions.
All algebraic and relational operations remain well-defined.

What is lost is not computability,
but unambiguous physical reference.

Beyond $\Sigma$, statements no longer admit
a unique relational preimage.

\subsection{No physical ontology}

The Schwarzgrenze is not a physical object,
not a horizon,
and not a singularity.

It is a property of the mapping $\Pi$,
not of the relational space $\mathcal{U}$.

Any interpretation of $\Sigma$
as a material or dynamical entity
constitutes a category error.

\subsection{Universality}

Schwarzgrenzen are not restricted
to specific physical scales.

Planck limits, causal horizons,
measurement cutoffs,
and resolution thresholds
are particular manifestations
of the same structural phenomenon.

\subsection{Function}

The Schwarzgrenze serves to:

\begin{itemize}
    \item mark the boundary of physical interpretability,
    \item separate computation from interpretation,
    \item prevent implicit category transitions.
\end{itemize}

Formal continuation beyond $\Sigma$
is explicitly permitted,
but carries no physical truth value.
