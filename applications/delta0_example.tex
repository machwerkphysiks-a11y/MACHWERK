\section{Delta Zero Example}

Let
\[
\Pi : \mathcal{U} \rightarrow \mathcal{M}
\]
be a projection into an observable calculation space.

\subsection*{Definition}

Define $\Delta_0$ as a formal decision marker indicating the loss of
physical interpretability while preserving formal definability.

\[
\Delta_0 \neq 0,\quad \Delta_0 \notin \mathbb{R}
\]

$\Delta_0$ is not a numerical value.

\subsection*{Operational Meaning}

A statement reaches $\Delta_0$ if:

\begin{itemize}
\item it remains syntactically well-defined,
\item it remains mathematically computable,
\item it is no longer uniquely back-projectable through $\Pi$.
\end{itemize}

\subsection*{Formal Criterion}

Let $S(u)$ be a statement depending on $u \in \mathcal{U}$.

If there exist
\[
u_1 \neq u_2
\quad\text{such that}\quad
\Pi(u_1)=\Pi(u_2)
\]
and
\[
S(u_1)\neq S(u_2),
\]
then the statement is marked by $\Delta_0$.

\subsection*{Relation to Schwarzgrenze}

$\Delta_0$ occurs at or beyond the Schwarzgrenze $\Sigma$.

While $\Sigma$ denotes structural non-injectivity,
$\Delta_0$ marks the point where a concrete computation
loses physical admissibility.

\subsection*{Role in MACHWERK}

$\Delta_0$ allows formal continuation
without assigning physical meaning.

It prevents mathematical artifacts
from being misinterpreted as physical results.
