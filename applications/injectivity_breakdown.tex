\section{Injectivity Breakdown}

This application demonstrates the explicit breakdown of injectivity
under a projection from the relational reference space into an observable
calculation domain.

\subsection{Setup}

Let
\[
\Pi : \mathcal{U} \rightarrow m_2
\]
be a projection from the relational full space $\mathcal{U}$ into a
stable observable domain $m_2$.

Assume two distinct relational configurations
\[
u_1 \neq u_2 \in \mathcal{U}
\]

\subsection{Projection Result}

Despite being distinct in $\mathcal{U}$, the projection yields
\[
\Pi(u_1) = \Pi(u_2)
\]

This equality is not the result of approximation or measurement error,
but a structural property of the projection.

\subsection{Injectivity Loss}

By definition, injectivity requires
\[
\forall u_1 \neq u_2:\ \Pi(u_1) \neq \Pi(u_2)
\]

This condition is violated.

Therefore, $\Pi$ is non-injective on the subset
\[
\Sigma \subset \mathcal{U}
\]

\subsection{Consequence}

Within $\Sigma$, observable states in $m_2$ no longer uniquely identify
their relational origin.

Any statement depending on a distinction between $u_1$ and $u_2$
cannot be physically admissible, even if it is mathematically well-defined.

\subsection{Classification}

This situation defines a Schwarzgrenze.

No physical inconsistency occurs.
Only interpretative uniqueness is lost.
