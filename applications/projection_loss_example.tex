\section{Projection Loss Example}

Let
\[
\Pi : \mathcal{U} \rightarrow \mathcal{M}
\]
be a projection from the relational full space into an observable calculation space.

Assume two distinct relational states
\[
u_1 \neq u_2 \in \mathcal{U}
\]
such that
\[
\Pi(u_1) = \Pi(u_2).
\]

\subsection*{Loss of Injectivity}

The projection $\Pi$ is not injective in this region.

Therefore, the observable image in $\mathcal{M}$ does not uniquely determine
the originating relational state in $\mathcal{U}$.

\subsection*{Consequences}

Any statement
\[
S(u)
\]
that distinguishes between $u_1$ and $u_2$ is not invariant under $\Pi$.

\subsection*{CRA Evaluation}

A statement is physically admissible only if:
\[
\Pi(u_1) = \Pi(u_2) \Rightarrow S(u_1) = S(u_2).
\]

This condition is violated here.

\subsection*{Result}

The computation remains formally valid.

The physical interpretation is inadmissible.

This marks the transition to the Schwarzgrenze.
