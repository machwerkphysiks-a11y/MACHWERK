\section{Delta Zero Application}

This application illustrates the use of $\Delta_0$
as a formal decision marker for interpretability.

\subsection{Setup}

Let
\[
\Pi : \mathcal{U} \rightarrow m_2
\]
be a projection into an observable calculation space.

Consider a formally well-defined statement
\[
S(u)
\quad\text{with}\quad u \in \mathcal{U}
\]

\subsection{Loss of Decidability}

Assume that $u$ lies in a non-injective region of $\Pi$.
Then multiple relational configurations correspond
to the same observable state.

\subsection{Delta Zero Condition}

If
\[
\Pi(u_1) = \Pi(u_2)
\quad\text{and}\quad
S(u_1) \neq S(u_2)
\]
then the statement $S$ is not decidable within $m_2$.

This condition is marked by
\[
S(u) \;\Rightarrow\; \Delta_0
\]

\subsection{Role of $\Delta_0$}

$\Delta_0$ indicates:

\begin{itemize}
\item the computation remains formally valid
\item physical interpretation is suspended
\item no additional assumptions are introduced
\end{itemize}

\subsection{Interpretation Rule}

$\Delta_0$ does not terminate calculations.
It marks the transition from interpretable
to non-interpretable results.

The model continues formally beyond $\Delta_0$,
without physical attribution.
