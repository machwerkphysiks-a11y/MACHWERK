\section{CRA Axiom}

\textbf{CRA Axiom (Consistency of Relational Applicability)}

A relational statement is physically admissible if and only if
all involved quantities

\begin{enumerate}
    \item are defined within the same relational validity domain,
    \item are subject to the same projection boundary,
    \item are combined without implicit category transitions.
\end{enumerate}

Let
\[
\mathcal{R} = \{ r_1, r_2, \dots, r_n \}
\]
be a finite set of relational expressions.

A statement
\[
S(\mathcal{R})
\]
is admissible if and only if there exists a single validity domain \( m_2 \)
such that
\[
\forall r_i \in \mathcal{R}, \quad r_i \in m_2.
\]

If no such domain exists, the statement is formally writable
but physically non-admissible.

Violation of the CRA axiom does not invalidate the mathematics.
It invalidates the physical interpretation.

The CRA axiom therefore separates
formal computability
from physical applicability.
